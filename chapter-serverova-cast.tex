\chapter{Server}
Jak již bylo zmíněno dříve, převážná část serveru byla napsána mým vedoucím Martinem Nečaským a sloužila i jako technická specifikace pro klientsou část. Server je napsaný v JavaScriptu a běží v Node.js\footnote{\url{https://nodejs.org/}}. Server je bezestavový a odpovídá na požadavky zmíněné dále.

\section{Jazyková podpora}
Server jsem částečně přepsal, aby podporoval dotazování na data z více světových jazyků. Některé požadavky přijímají parametr \texttt{languages} obsahující čárkou \texttt{,} oddělené ISO 639-1 jazykové kódy. Server pak vrací objekty jejiž klíčem je jazykový kód a hodnotou daný překlad do jazyku. Pokud překlad neexistuje, hodnotou je \texttt{undefined}. V případě, že na všechny jazyky bylo vráceno \texttt{undefined}, server se pokusí přidat další jazyk, který existuje. Který jazyk takto bude vybrán není určeno.

\begin{prikl}
Pro \texttt{languages=cs,en} může server vrátit například
\begin{code}[frame=none]
{
    cs: undefined,
    en: "Kankakee County"
}
\end{code}
ale pokud nezná překlad ani do češtiny, ani do angličity, může vrátit
\begin{code}[frame=none]
{
    cs: undefined,
    en: undefined,
    sk: "Jazero Beňatina"
}
\end{code}
\end{prikl}

\section{API}
Server vrací data ve formátu JSON, požadavky jsou posílány metodou GET a parametry jsou kódovány do URL adresy.

\begin{itemize}
    \item \texttt{/metaconfiguration}, \textbf{parametry:} \texttt{iri} a \texttt{languages} \\
    Vrátí informace o metakonfiguraci zadané podle \texttt{iri}. \\
    Vrátí všechna data (viz kapitola \ref{pozadavky-metakonfigurace}) o metakonfiguraci, veškerá data o dceřiných konfiguracích (viz kapitola \ref{pozadavky-konfigurace}) a základní data o dceřiných metakonfiguracích (vše kromě seznamu konfigurací a metakonfigurací).

    \item \texttt{/configuration}, \textbf{parametry:} \texttt{iri} a \texttt{languages} \\
    Vrátí informace o konfiguraci zadané podle \texttt{iri}. \\
    Vrátí stejná data jako \texttt{/metaconfiguration} o svých sceřiných konfiguracích. \\
    Toto volání se používá pouze když uživatel ručně zvolí IRI konfigurace, v opačném případě si aplikace vystačí s voláním \texttt{/metaconfiguration}.

    \item \texttt{/stylesheet}, \textbf{parametry:} \texttt{stylesheet} \\
    Vrátí kompletní visual style sheet (viz kapitola \ref{pozadavky-visual-style-sheet}) na základě jeho IRI jako parametr \texttt{stylesheet}.

    \item \texttt{/view-sets}, \textbf{parametry:} \texttt{config} a \texttt{resource} \\
    Vrátí seznam možných view setů (viz kapitola \ref{pozadavky-view-sets}) které odpovídají vrcholu s IRI \texttt{resource} při dané konfiguraci \texttt{config}.

    \item \texttt{/preview}, \textbf{parametry:} \texttt{view} a \texttt{resource} \\
    Vrátí data z dotazu preview (viz kapitola \ref{pozadavky-preview}) na vrchol s IRI \texttt{resource} při daném pohledu \texttt{view}.

    \item \texttt{/detail}, \textbf{parametry:} \texttt{view} a \texttt{resource} \\
    Vrátí data z dotazu detail (viz kapitola \ref{pozadavky-detail}) na vrchol s IRI \texttt{resource} při daném pohledu \texttt{view}.

    \item \texttt{/expand}, \textbf{parametry:} \texttt{view} a \texttt{resource} \\
    Vrátí expandované vrcholy (viz kapitola \ref{pozadavky-expansion}) pro vrchol s IRI \texttt{resource} při daném pohledu \texttt{view}. Tyto expandované vrcholy již obsahují data o detailu a tedy není třeba žádného dalšího volání.
\end{itemize}