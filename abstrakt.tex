%%% Šablona pro jednoduchý soubor formátu PDF/A, jako treba samostatný abstrakt práce.

\documentclass[12pt]{report}

\usepackage[a4paper, hmargin=1in, vmargin=1in]{geometry}
\usepackage[a-2u]{pdfx}
\usepackage[czech]{babel}
\usepackage[utf8]{inputenc}
\usepackage[T1]{fontenc}
\usepackage{lmodern}
\usepackage{textcomp}

\begin{document}

%% Nezapomeňte upravit abstrakt.xmpdata.

Jeden ze způsobů, jak publikovat ve strojově čitelné podobě data na internetu, je formou grafu. Taková data je pak velmi snadné propojovat mezi různými datovými zdroji a vytvořit tak velkou síť propojených dat. Abychom tyto data mohli vizualizovat, musíme používat nástroje na procházení grafu, které mohou být často nepraktické, neboť nám obvykle zobrazí veškeré informace, kterých mohou být stovky.

Cílem této práce je vytvořit webovou aplikaci, která je schopna tyto grafová data vizualizovat, poskytovat k nim informace a procházet je pomocí takzvaných konfigurací. Konfigurace popisuje, jaké data z velké množiny dat na internetu chceme zobrazit, jak je vizualizujeme a jak se na ně můžeme dívat. Dovolí nám tedy odfiltrovat stovky nezajímavých informací a odstíní tak uživatele od složité sítě propojených dat, aby se mohl věnovat pouze těm, která ho zajímají.

\end{document}
